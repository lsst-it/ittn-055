\section{Definitions}

The following concepts will be utilized during the development of this document

\subsection{Data Recovery}

\begin{itemize}
    \item \textbf{Recovery Point Objective}: Maximum temporal point in the history of the data, that can be lost. For example, user data that it is modified frequently could have a RPO of hours, but data less critical could have a RPO of days. 
    This metric is about length of time, it doesn't specify amount of data or quality of the lost data.
    
    \item \textbf{Recovery Time Objective}: Maximum length of time that a system can be offline. 
  \end{itemize}


\subsection{Recovery Status}

    \begin{itemize}
        \item \textbf{Cold Recovery}: There's no spare available and a replacement must be obtained from the vendor. Considerable downtime or degradation of the service is expected until the system is fixed.
        \item \textbf{Warm Recovery}: There is a spare onsite but it is not online. Maintenance window is required to recover the system.
        \item \textbf{Hot Recovery}: There is a spare onsite and it is online. No downtime is required to recover the system, but there could be degradation of the service
    \end{itemize}

