\documentclass[PMO,authoryear,lsstdraft,toc]{lsstdoc}
% lsstdoc documentation: https://lsst-texmf.lsst.io/lsstdoc.html
% \input{meta}

% Package imports go here.

% Local commands go here.

%If you want glossaries
%\input{aglossary.tex}
%\makeglossaries

\title{Disaster Recovery}

% Optional subtitle
% \setDocSubtitle{A subtitle}

\author{%
Cristian Silva
}

\setDocRef{ITTN-055}
\setDocUpstreamLocation{\url{https://github.com/lsst-it/ittn-055}}

% \date{\vcsDate}
\date {\today}

% Optional: name of the document's curator
% \setDocCurator{The Curator of this Document}

\setDocAbstract{%
Details of IT disaster recovery procedures. 
}

% Change history defined here.
% Order: oldest first.
% Fields: VERSION, DATE, DESCRIPTION, OWNER NAME.
% See LPM-51 for version number policy.
\setDocChangeRecord{%
  \addtohist{1}{2022-03-24}{Unreleased.}{Cristian Silva}
  \addtohist{2}{2023-11-24}{Early Draft}{Cristian Silva}
 
}


\begin{document}

% Create the title page.
\maketitle
% Frequently for a technote we do not want a title page  uncomment this to remove the title page and changelog.
% use \mkshorttitle to remove the extra pages

% ADD CONTENT HERE
% You can also use the \input command to include several content files.
\section{Introduction}

Within the landscape of modern technology, unforeseen disruptions poses a significant challenge. Recognizing the importance of preserving data integrity, system functionality, and business continuity, this Disaster Recovery document outlines a comprehensive strategy. Rooted in best practices and planning, this framework aims to fortify our organization's resilience, ensuring a swift and effective response to any disruptive event
\section{Scope}

This disaster recovery document encompasses the recovery procedures and strategies applicable to the core infrastructure of the Vera C. Rubin Observatory. 
The scope includes but is not limited to:

\subsection{Hardware Systems}

Servers hosting critical applications and databases.
Network devices and communication infrastructure.
End-user computing devices essential for business operations.

\subsection{Software Systems}

Operating systems and system software.
Business applications crucial for day-to-day operations.
Database management systems storing mission-critical data.

\subsection{Data Management}

Backup and recovery procedures for ensuring data integrity.
Procedures for the restoration of databases and file systems.

\subsection{Network Infrastructure}

Communication networks connecting internal and external systems.
Security appliances and protocols safeguarding network integrity.


\section{Computing Resources}

Details of the Computing Resources plan can be found in technote \href{http://ittn-057.lsst.io}{ittn-057}

\section{Network Devices}

Details of the Network Devices plan can be found in technote \href{http://ittn-056.lsst.io}{ittn-056}

\section{Infrastructure Support Devices}

Details of the Infrastructure Support Devices plan can be found in technote \href{http://ittn-058.lsst.io}{ittn-058}


\appendix
% Include all the relevant bib files.
% https://lsst-texmf.lsst.io/lsstdoc.html#bibliographies
\section{References} \label{sec:bib}
\renewcommand{\refname}{} % Suppress default Bibliography section
\bibliography{local,lsst,lsst-dm,refs_ads,refs,books}

% Make sure lsst-texmf/bin/generateAcronyms.py is in your path
\section{Acronyms} \label{sec:acronyms}
\input{acronyms.tex}
% If you want glossary uncomment below -- comment out the two lines above
%\printglossaries





\end{document}
